
\section{Centered n-gons}
Consider an n-gon with corners ${1,...,n}$ and a center point $0$. Now consider the simplicial complex consisting of the vertices $\{(1),...,(n)\}$, the edges $\{(0,1),...,(0,n),(1,2),...,(n-1,n),(1,n)\}$ and the triangles \\$\{(0,1,2),...,(0,n-1,n),(0,1,n)\}$. \\The following figure gives an example for a pentagon with a center point.

\PD{Will we get different results if the edegs in the external boundary are paired in the same way as the triangles?}

\begin{figure}[H]
\noindent%
\centering%
\fbox{%
\begin{minipage}[c][4.9cm][c]{\dimexpr.49\linewidth-2\fboxsep-2\fboxrule}%
\centering%
\begin{tikzpicture}
\node[b_circle, name path=0, label=below:{0}] (0) at (0,0.25) {};

\node[b_circle, name path=1, label=left:{1}] (1) at (-1.5,1) {};

\node[b_circle, name path=2, label=above:{2}] (2) at  (0,2) {};

\node[b_circle, name path=3, label=right:{3}] (3) at  (1.5,1) {};

\node[b_circle, name path=4, label=below:{4}] (4) at (1,-1) {};

\node[b_circle, name path=5, label=below:{5}] (5) at  (-1,-1) {};

\foreach \i in {1,...,5}
{
    \draw[thick] (0) to (\i);
}

\draw[thick] (1) to (2) to (3) to (4) to (5) to (1);

\fill[opacity=0.2] (-1.5,1)--(0,2)--(1.5,1)--(1,-1)--(-1,-1)--(-1.5,1);


%\draw [-,very thick] (2,0,2) -- (0.25,0,1);

\end{tikzpicture}


\end{minipage}}%
%\captionsetup{width=0.9\textwidth}
\caption{Triangulation of a pentagon with a central point.}
\label{fig:centeredpentagon}
\end{figure}

\begin{lemma}[Quadratic number of additions]
Let $K$ be a simplicial complex of an n-gon with center, as defined above. The filtration: 
\begin{center}
    $\{(n),...,(0),(n-1,n),(n-2,n-1),...,(1,2),(0,1),(0,2),...,(0,n),(1,n),...\}$
\end{center}
yields a boundarymatrix, for which the column reduction algorithm needs more than $\sum_{i=1}^n i = \frac{n(n+1)}{2} \in \Omega(n^2)$ additions.
\end{lemma}
\begin{proof}
Coming soon.
\end{proof}

The following figure shows the plot of the number of simplices in the n-gon with center mapped to the number of additions needed by the column reduction algorithm. Note that the number of simplices equals $1+4n$.


\begin{figure}[H]
\noindent%
\centering%
\centering%
\includegraphics[width=\linewidth]{ConnectingMorsePersistence/Figures/n_gon_plot_many_additions.png}

%\captionsetup{width=0.9\textwidth}
\caption{Number of additions of the reduction algorithm}
\label{fig:quadratic_plot
}
\end{figure}

Let $F$ be a filtration as in Lemma 3.1. for $n = 5$ The following figure shows a the output of the algorithm Filtration\_to\_DMF($F$).

\begin{figure}[H]
\noindent%
\centering%
\fbox{%
\begin{minipage}[c][4.9cm][c]{\dimexpr.49\linewidth-2\fboxsep-2\fboxrule}%
\centering%
\begin{tikzpicture}
\node[b_circle, name path=0, label=below:{0}] (0) at (0,0.25) {};

\node[b_circle, name path=1, label=left:{1}] (1) at (-1.5,1) {};

\node[b_circle, name path=2, label=above:{2}] (2) at  (0,2) {};

\node[b_circle, name path=3, label=right:{3}] (3) at  (1.5,1) {};

\node[b_circle, name path=4, label=below:{4}] (4) at (1,-1) {};

\node[b_circle, name path=5, label=below:{5}] (5) at  (-1,-1) {};

\foreach \i in {2,...,5}
{
    \draw[thick] (0) to (\i);
}

\draw[thick] (1) to (5);

\draw[->-, thick] (0) to (1);
\draw[->-, thick] (1) to (2);
\draw[->-, thick] (2) to (3);
\draw[->-, thick] (3) to (4);
\draw[->-, thick] (4) to (5);

\node[inner sep = 0] (6) at ($(1)!0.5!(5)$) {};
\node[inner sep = 0] (7) at ($(0)!0.5!(5)$) {};
\node[inner sep = 0] (8) at ($(0)!0.5!(4)$) {};
\node[inner sep = 0] (9) at ($(0)!0.5!(3)$) {};
\node[inner sep = 0] (10) at ($(0)!0.5!(2)$) {};
\node[inner sep = 0] (11) at ($(0)!0.5!(1)$) {};

\draw[->, thick] (6) to ($(6)!0.3!(0)$);

\draw[->, thick] (7) to ($(7)!0.3!(8)$);

\draw[->, thick] (8) to ($(8)!0.3!(9)$);

\draw[->, thick] (9) to ($(9)!0.3!(10)$);

\draw[->, thick] (10) to ($(10)!0.3!(11)$);

\fill[opacity=0.2] (-1.5,1)--(0,2)--(1.5,1)--(1,-1)--(-1,-1)--(-1.5,1);


%\draw [-,very thick] (2,0,2) -- (0.25,0,1);

\end{tikzpicture}


\end{minipage}}%
%\captionsetup{width=0.9\textwidth}
\caption{Triangulation of a pentagon with a central point.}
\label{fig:pentagon_quad_morse}
\end{figure}


\section{Conflicts along v-paths}
\begin{lemma}[Conflictfree 2-dim v-path]
Let $K$ be a simplicial complex and $V$ a gradient field on $K$. Let $p = (v_1,e_1,v_2,...,e_{n-1},v_n)$ be a v-path of vertices and edges in $V$. Let $F$ be the filtration given by the algorithm \textbf{DMF\_to\_Filtration}. Let $B_p$ be the submatrix of the boundary Matrix of $F$, that only consists of rows $v_1,...,v_n$ and columns $e_1,...,e_{n-1}$. It holds that $\low(i) = i+1$.
\end{lemma}

\begin{proof}
Note that in the full boundary matrix there might be columns or rows in between the rows and columns of $B_p$, but we will ignore those. \\
The v-path $p$ enforces that $v_1,...,v_n$ appear in that order. So $v_1$ before $v_2$ before $v_3$ and so on. The same holds for $e_1,...,e_{n-1}$. This means the $i$-th column, for $i \in \{i,...,n-1\}$ has non zero entries in the rows corresponding to $v_i$ and $v_{i+1}$, since the simplex $e_i$ is adjacent to the simplices $v_i$ and $v_{i+1}$. Since $v_{i+1}$ is the $(i+1)$-th row, the statement follows.
\end{proof}

\PD{I do not clearly see from that proof if there are no intermediate cells, of indices below the maximal index of an element in the V-Path, that do not creat conflicts. I do see that there will not be clnfilicts between elements of v-path, but what about elements of v-path and other simplices?}